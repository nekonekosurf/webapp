\documentclass[a4j]{jarticle}
\usepackage{listings}
\usepackage{color}
\definecolor{OliveGreen}{rgb}{0.0,0.6,0.0}
\lstset{
  basicstyle={\ttfamily},
  identifierstyle={\small},
  commentstyle={\smallitshape \color[rgb]{0,0.5,0}},
  keywordstyle={\small\bfseries},
  ndkeywordstyle={\small},
  stringstyle={\small\ttfamily},
  frame={tb},
  breaklines=true,
  columns=[l]{fullflexible},
  numbers=left,
  xrightmargin=0zw,
  xleftmargin=0zw,
  numberstyle={\scriptsize},
  stepnumber=1,
  numbersep=1zw,
  lineskip=-0.5ex,
  linewidth=15cm
}
%ここまでソースコードの表示に関する設定
\begin{document}
以下使用したソースコードを示す.

\lstinputlisting[caption = ex110.html]{../webapp/1stWeek/ex110.html}
\newpage
\lstinputlisting[caption = ex120.html]{../webapp/1stWeek/ex120.html}
\newpage
\lstinputlisting[caption = ex210.html]{../webapp/1stWeek/ex210.html}
\newpage
\lstinputlisting[caption = ex220.html]{../webapp/1stWeek/ex220.html}
\newpage
\lstinputlisting[caption =ex410short.html ]{../webapp/2ndWeek/ex410short.html}
\newpage
\lstinputlisting[caption =ex421long.html ]{../webapp/2ndWeek/ex421long.html}
\newpage
\lstinputlisting[caption = ex430.html]{../webapp/2ndWeek/ex430.html}
\newpage
\lstinputlisting[caption = ex510dev.html]{../webapp/2ndWeek/ex510dev.html}
\newpage
\lstinputlisting[caption = ex520\_and\_521.html]{../webapp/2ndWeek/ex520_and_521.html}
\newpage
% \begin{lstlisting}[caption=ex110.html]
%   <!DOCTYPE HTML>
%   <html>
%     <head>
%       <meta charset="UTF-8">
%       <title>ex220: テキストの行を追加・削除する</title>
%       <script>
%         var i = 0;
%         var nodeArray = new Array(100);
%         function addText(event){
%       		i++;
%       		node = document.createElement('p');
%           document.body.appendChild(node);
%     		  node.innerHTML =i+'テキストが追加されました';
%     		  nodeArray[i] =node ;
%   	    }
%         function removeText(event){
%           document.body.removeChild(nodeArray[i])
%           i--;
%         }
%         window.addEventListener("load", function(event){
%         var addButton    = document.getElementById("add-text");
%         var removeButton = document.getElementById("remove-text");
%         addButton.   addEventListener("click", addText,    false);
%         removeButton.addEventListener("click", removeText, false);
%         }, false);
%       </script>
%     </head>
%     <body>
%       <div id="box">
%         <button id="add-text">A</button>
%         <button id="remove-text">B</button>
%       </div>
%     </body>
%   </html>
% \end{lstlisting}
% \newpage
%
% \begin{lstlisting}[caption=ex120.html]
%   <!-- 16進数(1~F)の掛け算表を表示するプログラム -->
%   <!DOCTYPE HTML>
%   <html>
%   <head>
%   <meta charset="UTF-8">
%   <title>ex120: 16進数の掛け算表</title>
%   </head>
%   <body>
%   <script>
%   function to_hex(decimal) {
%     if (decimal>=0 && decimal<=255){
%       var re =decimal.toString(16) ;
%       return re
%     }
%     else{
%       return 0;
%     }
%   }
%   var n = 16;	   //二次元配列の行、列の要素数(16×16の配列を作る)
%   var tableData;  //掛け算表のデータを格納するn×nの二次元配列データ
%   tableData = new Array(n)
%   for (var i =0;i<n;i++){
%     tableData[i]= new Array(n);
%   }
%   for (var i=0;i<n;i++){
%     for (var j=0;j<n;j++){
%       tableData[i][j]= to_hex(i*j);
%       if(i==0){
%         tableData[i][j]=j;
%       }
%       if(j==0){
%         tableData[i][j]=i;
%       }
%       if (j==0 && i==0){ //こいつの位置は固定
%         tableData[i][i]="\\ "	;
%       }
%     }
%   }
%   document.write("<table border>");  //表の定義を開始、以降のコードで表の中身を定義していく
%   for(var i=0; i<n; i++){
%     document.write("<tr>");  //表の横一行部分の定義を開始
%     for(var j=0; j<n; j++){
%       //以下のコードを変更し、表の各セルに掛け算表のデータを出力するコードをここに書くこと。
%       document.write("<td> ",tableData[i][j], " </td>");
%     }
%     document.write("</tr>");  //表の横一行部分の定義を終了
%   }
%   document.write("</table>");  //表の定義を終了
%   </script>
%   </body>
%   </html>
% \end{lstlisting}
%
% \newpage
% \begin{lstlisting}[caption=ex210.html]
%   <!-- 16進数(1~F)の掛け算表を表示するプログラム -->
%   <!DOCTYPE HTML>
%   <html>
%   <head>
%   <meta charset="UTF-8">
%   <title>ex120: 16進数の掛け算表</title>
%   </head>
%   <body>
%   <script>
%   function to_hex(decimal) {
%     if (decimal>=0 && decimal<=255){
%       var re =decimal.toString(16) ;
%       return re
%     }
%     else{
%       return 0;
%     }
%   }
%   var n = 16;	   //二次元配列の行、列の要素数(16×16の配列を作る)
%   var tableData;  //掛け算表のデータを格納するn×nの二次元配列データ
%   tableData = new Array(n)
%   for (var i =0;i<n;i++){
%     tableData[i]= new Array(n);
%   }
%   for (var i=0;i<n;i++){
%     for (var j=0;j<n;j++){
%       tableData[i][j]= to_hex(i*j);
%
%       if(i==0){
%         tableData[i][j]=j;
%       }
%       if(j==0){
%         tableData[i][j]=i;
%       }
%       if (j==0 && i==0){ //こいつの位置は固定
%         tableData[i][i]="\\ "	;
%       }
%     }
%   }
%   document.write("<table border>");  //表の定義を開始、以降のコードで表の中身を定義していく
%   for(var i=0; i<n; i++){
%     document.write("<tr>");  //表の横一行部分の定義を開始
%     for(var j=0; j<n; j++){
%       document.write("<td> ",tableData[i][j], " </td>");
%     }
%
%     document.write("</tr>");  //表の横一行部分の定義を終了
%   }
%   document.write("</table>");  //表の定義を終了
%   </script>
%   </body>
%   </html>
% \end{lstlisting}
% \newpage
%
% \begin{lstlisting}[caption=ex220.html]
%   <!DOCTYPE HTML>
%   <html>
%   <head>
%   <meta charset="UTF-8">
%   <title>ex220: テキストの行を追加・削除する</title>
%   <script>
%   //現在の行数を表す変数、初期値は0
%   var i = 0;
%   var nodeArray = new Array(100);
%   function addText(event){
%     i++;
%     node = document.createElement('p');
%     node.innerHTML =i+'テキストが追加されました';
%     document.body.appendChild(node);
%     nodeArray[i] =node ;
%   }
%   function removeText(event){
%     document.body.removeChild(nodeArray[i])
%     i--;
%   }
%   window.addEventListener("load", function(event){
%     var addButton    = document.getElementById("add-text");
%     var removeButton = document.getElementById("remove-text");
%     addButton.   addEventListener("click", addText,    false);
%     removeButton.addEventListener("click", removeText, false);
%     }, false);
%     </script>
%     </head>
%     <body>
%     <div id="box">
%     <button id="add-text">A</button>
%     <button id="remove-text">B</button>
%     </div>
%     </body>
%     </html>
% \end{lstlisting}
% \newpage
%
% \begin{lstlisting}[caption=ex410short.html]
%   <!-- 省略しない記法で書かれたプログラムを省略記法に書き直す課題 -->
%   <!DOCTYPE HTML>
%   <html>
%     <head>
%       <meta charset="UTF-8">
%       <title>ex410: ajaxの実験</title>
%       <!-- jQueryライブラリの指定:2019/10/31現在最新版をダウンロード -->
%       <script src="https://ajax.googleapis.com/ajax/libs/jquery/3.4.1/jquery.min.js"></script>
%       <script>
%         var left_state  = false;
%         var right_state = false;
%         $(function(){
%           $('#button1').click(
%             function(){
%               if(left_state==false){
%                 $.ajax('ex410-load.txt').done(
%                   function(data){
%                     $('#status-area1').text('読み込み成功');
%                     $('#display-area1').text(data);
%                   }
%                 ).fail(
%                   function(data){
%                     $('#status-area1').text('読み込み失敗');
%                   }
%                 );
%                 left_state=true;
%               }else{
%                 $('#status-area1').text('');
%                 $('#display-area1').text('');
%                 left_state = false;
%               }
%             }
%           );
%           $('#button2').click(
%             function(){
%               if(right_state==false){
%                 $.ajax('ex410-load.html').done(
%                   function(data){
%                     $('#status-area2').text('読み込み成功');
%                     $('#display-area2').html(data);
%                   }
%                 ).fail(
%                   function(data){
%                     $('#status-area2').text('読み込み失敗');
%                   }
%                 );
%                 right_state=true;
%               }else{
%                 $('#status-area2').text('');
%                 $('#display-area2').html('');
%                 right_state = false;
%               }
%             }
%           );
%         }
%       );
%       </script>
%     </head>
%
%     <body bgcolor="#FFFFFF">
%       <table>
%         <tr>
%           <td>
%             <div style='clear:both;border: 2px dotted #080;margin:10px; padding:10px;'>
%               <p>
%                 <button id='button1'>Ajax で取得したテキストファイルを表示</button>
%               </p>
%               <span id='status-area1'></span><br>
%               <div id='display-area1' style='border:1px solid #00f;padding:10px;font-size:20px;font-weight:bold; height: 100px'></div>
%             </div>
%           </td>
%           <td>
%             <div style='clear:both;border: 2px dotted #080;margin:10px; padding:10px;'>
%               <p>
%                 <button id='button2'>Ajax で取得した HTML を表示</button>
%               </p>
%               <span id='status-area2'></span><br>
%               <div id='display-area2' style='border:1px solid #00f;padding:10px;font-size:20px;font-weight:bold; height: 100px'></div>
%             </div>
%           </td>
%         </tr>
%       </table>
%     </body>
%   </html>
% \end{lstlisting}
% \newpage
%
% \begin{lstlisting}[caption=ex421long.html]
%   <!DOCTYPE HTML>
%   <html>
%
%     <head>
%       <meta charset="UTF-8">
%       <title>ex421: 天気情報取得</title>
%       <script src="https://code.jquery.com/jquery-3.4.1.min.js"></script>
%       <script src="./mamewaza/mamewaza_weather.min.js"></script>
%       <script>
%       var state = false;
%       jQuery(document).ready(init);
%       function init(){
%         jQuery('#okinawa').click(okinawa_weather);
%         jQuery('#oosaka').click(oosaka_weather);
%         jQuery('#tokyo').click(tokyo_wewather);
%       }
%       function oosaka_weather(){
%         if(state == true){
%           state = false;
%           jQuery('#city').text('');
%         }
%         jQuery('#name').text('大坂の天気')
%         mamewaza_func(270000)//大坂
%         state = true;
%       }
%
%       function okinawa_weather(){
%         if(state == true){
%           state = false;
%           jQuery('#city').text('');
%         }
%         jQuery('#name').text('那覇の天気')
%         mamewaza_func(471010)//那覇
%         state= true;
%       }
%       function tokyo_wewather(){
%         if(state == true){
%           state = false;
%           jQuery('#city').text('');
%         }
%         jQuery('#name').text('東京の天気')
%         mamewaza_func(130010)//東京
%         state = true;
%       }
%       function mamewaza_func(city){
%         jQuery.mamewaza_weather({
%           selector: "#city",
%           region: city,//東京に変更 仕様(1)
%           layout: "horizontal",//表示形式変更 仕様(2)
%           when: "2days",//表示形式変更 仕様(2)
%           explanation: true,//説明表示 仕様(3)
%           cssPath: "./mamewaza/mamewaza_weather.css"
%         })
%       }
%       </script>
%     </head>
%     <body>
%       <div>
%         <p>ボタンを押して都市を選択してください</p>
%         <span id='name'></span><br>
%         <p>
%           <button id='okinawa'>沖縄</button>
%           <button id='oosaka'>大坂</button>
%           <button id='tokyo'>東京</button
%         </p>
%         <span id='city'></span><br>
%         <!-- <p id="city"></p> -->
%       </div>
%     </body>
%   </html>
% \end{lstlisting}
% \newpage
%
% \begin{lstlisting}[caption=ex430.html]
%   <!-- jQueryを利用したアニメーションの課題 -->
%   <!DOCTYPE HTML>
%   <html>
%     <head>
%       <meta charset="UTF-8">
%       <title>ex430: パラパラアニメーション</title>
%       <!-- jQueryライブラリの指定:2019/10/31現在最新版をダウンロード -->
%       <script src="https://ajax.googleapis.com/ajax/libs/jquery/3.4.1/jquery.min.js"></script>
%
%       <style type="text/css"> /*style内の書式はcss*/
%         /*myImageというID属性が付けられたHTML要素に対してスタイルを指定*/
%         #myImage {
%          	position: absolute;        /*配置する位置を絶対位置で指定
%             (absoluteの場合はブラウザウインドウの左上が基準位置(0,0)となる)*/
%           top: 20px;                 /*ウインドウ左上から20px下に配置*/
%           left: 100px;               /*ウインドウ左上から100px右に配置*/
%           -moz-user-select: none;    /*ダブルクリックやドラッグで選択されない
%                                        ようにする(Firefox版)*/
%           -webkit-user-select: none; /*(Safari, Chrome版)*/
%         }
%       </style>
%       <script type="text/javascript">
%         var NumOfImage = 4;
%         var INTERVAL = 200;
%         var currentImg = 0;
%         var timerId;
%         var isAnimating = false;
%         var stopped = false;
%         $(function(){
%           for(var i=0; i<NumOfImage; i++)
%             $("<img>").attr("src", "images/img" + i + ".png");
%           $("<img id='myImage' width='80' height='120'>")
%             .attr("src", "images/img0.png")
%             .appendTo("#myDiv");
%           $("#startBtn").click(function(){
%             //パラパラアニメーション中なら終了させる
%             if (isAnimating) {
%               clearTimeout(timerId);
%               isAnimating = false;
%               $("#startBtn").val("スタート");
%             //そうでなければアニメーション開始
%             }else{
%               isAnimating = true;
%               flipAnimate();
%               $("#startBtn").val("ストップ");
%             }
%           });
%         });
%         $(document).dblclick(function(event){
%           if (isAnimating) { //再生中
%             clearTimeout(timerId);
%             isAnimating = true;
%             $("#startBtn").val("ストップ");
%     				$("#myImage").animate({
%     											"left": event.pageX-$("#myImage").width()/2,
%     											//"left": event.pageX,
%     											"top": event.pageY-$("#myImage").height()/2}
%     											//"top": event.pageY}
%     											,"normal",flipAnimate)
%   					}	else{//停止中
%   						$("#myImage").animate({
%   															"left": event.pageX-$("#myImage").width()/2,
%   															//"left": event.pageX,
%   															"top": event.pageY-$("#myImage").height()/2}
%   															//"top": event.pageY}
%   															,"normal")
%   					}
%         });
%         function flipAnimate() {
%           $("#myImage").attr("src", "images/img" + currentImg + ".png");
%           currentImg++;  //次に表示したい画像の番号を設定(ローテーションにする)
%           if(currentImg >= NumOfImage) currentImg = 0;
%
%           //INTERVALで指定したミリ秒後に再びflipAnimate()関数を呼び出す
%         	timerId = setTimeout("flipAnimate()", INTERVAL);
%         }
%       </script>
%     </head>
%
%     <body>
%       <form name="form1"><p>
%           <input type="button" id="startBtn" value="スタート">
%       </p></form>
%       <div id="myDiv"></div>
%     </body>
%   </html>
%
% \end{lstlisting}
% \newpage
%
% \begin{lstlisting}[caption=ex510dev.html]
%   <!DOCTYPE html>
%   <html lang="ja">
%     <head>
%       <meta charset="UTF-8">
%       <title>自作したフォームページのサンプル</title>
%       <meta name="viewport" content="width=device-width,initial-scale=1">
%
%       <!-- 2019年10月31日現在のjQuery Mobileの最新安定板は1.4.5 -->
%       <link rel="stylesheet" href="https://ajax.googleapis.com/ajax/libs/jquerymobile/1.4.5/jquery.mobile.min.css" />
%
%       <!-- jQuery Mobileに対応している最新版 jQuery (2.1.4) -->
%       <script src="https://ajax.googleapis.com/ajax/libs/jquery/2.1.4/jquery.min.js"></script>
%
%       <!-- 2019年10月31日現在のjQuery Mobileの最新安定板は1.4.5 -->
%       <script src="https://ajax.googleapis.com/ajax/libs/jquerymobile/1.4.5/jquery.mobile.min.js"></script>
%     </head>
%     <body>
%       <div data-role="page">
%         <div data-role="header">
%           <h1>注文フォーム</h1>
%         </div>
%         <div data-role="content">
%           <form id="inquiry_form" method="post">
%             <div data-role="fieldcontain">
%               <label for="name">お名前</label>
%               <input type="text" name="name" id="name" value="" />
%               <fieldset>
%             </div>
%             <div data-role="fieldcontain">
%               <label for="kind">商品名</label>
%               <select name="kind" id="kind">
%                 <option>選択してください</option>
%                 <option value="選択項目1">抹茶ラテ</option>
%                 <option value="選択項目2">コーヒーラテ2</option>
%                 <option value="選択項目3">ミルクティー</option>
%                 <option value="選択項目4">抹茶シェイク</option>
%               </select>
%             </div>
%             <div data-role="fieldcontain">
%               <fieldset data-role="controlgroup">
%                 <legend>甘さ</legend>
%                 <input type="radio" name="radio-choice" id="radio-choice-1" value="choise-1" />
%                 <label for="radio-choice-1">0%</label>
%                 <input type="radio" name="radio-choice" id="radio-choice-2" value="choise-2" />
%                 <label for="radio-choice-2">50%</label>
%                 <input type="radio" name="radio-choice" id="radio-choice-3" value="choise-3" />
%                 <label for="radio-choice-3">100%</label>
%               </fieldset>
%             </div>
%             <div data-role="fieldcontain">
%               <label for="slider">氷の量</label>
%               <span id = "value">少なめ</span>
%               <input type="range" name="slider" id="slider" value="1" min="0" max="2" step="1" onchange="showValue()"/>
%             </div>
%           <script>
%             function  showValue(){
%                if( document.getElementById('slider').value==0){
%                  document.getElementById('value').innerHTML = "なし";
%                }else if (document.getElementById('slider').value==0) {
%                  document.getElementById('value').innerHTML = "少なめ";
%                }else {
%                  document.getElementById('value').innerHTML = "普通";
%                }
%             }
%           </script>
%             <div data-role="fieldcontain">
%               <label for="flip-1">タピオ有無</label>
%               <select name="flip-1" id="flip-1" data-role="slider">
%                 <option value="off">入れる</option>
%                 <option value="on">入れない</option>
%               </select>
%             </div>
%             <div data-role="fieldcontain">
%               <label for="inquiery">備考</label>
%               <textarea name="inquiery" id="inquirery"></textarea>
%             </div>
%             <input type="submit" id="submit" data-thema="b" value="送信">
%           </form>
%         </div>
%       </div>
%     </body>
%   </html>
% \end{lstlisting}
% \newpage
%
% \begin{lstlisting}[caption=ex520\_and\_521.html]
%   <!DOCTYPE html>
%   <html>
%     <head>
%       <meta charset="UTF-8">
%       <title>自己開発バージョン</title>
%       <meta name="viewport" content="width=device-width, initial-scale=1">
%       <!-- 2018年10月31日現在のjQuery Mobileの最新安定板は1.4.5 -->
%       <link rel="stylesheet" href="https://ajax.googleapis.com/ajax/libs/jquerymobile/1.4.5/jquery.mobile.min.css" />
%
%       <!-- jQuery version 1の最新版を導入する書き方 -->
%       <script src="https://ajax.googleapis.com/ajax/libs/jquery/1.12.4/jquery.min.js"></script>
%
%       <!-- 2018年10月31日現在のjQuery Mobileの最新安定板は1.4.5 -->
%       <script src="https://ajax.googleapis.com/ajax/libs/jquerymobile/1.4.5/jquery.mobile.min.js"></script>
%       <script>
%         $(document).on('pagecreate', '#page1', function() {
%           $('#page1swipe').on('swipeleft', page2view);
%         });
%         $(document).on('pagecreate', '#page2', function() {
%            $('#page2swipe').on('swiperight', page1view);
%            $('#page2swipe').on('swipeleft', page3view);
%         });
%         $(document).on('pagecreate', '#page3', function() {
%            $('#page3swipe').on('swiperight', page2view);
%         });
%         // ページ1(#page1)へ進む
%         function page1view() {
%           $.mobile.changePage('#page1');
%         }
%         // ページ2(#page2)へ進む
%         function page2view() {
%           $.mobile.changePage('#page2');
%         }
%         // ページ3(#page3)へ進む
%         function page3view() {
%           $.mobile.changePage('#page3');
%         }
%       </script>
%     </head>
%     <body>
%       <!-- ページ1 -->
%       <div data-role="page" id="page1" data-theme="a">
%         <div data-role="header">
%           <!-- ナビゲーションバー(左右ページへのリンク) -->
%           <div data-role="navbar" class="navbar">
%             <ul>
%               <li>
%                 <a href="#page2" data-icon="arrow-r"></a>
%               </li>
%             </ul>
%           </div>
%         </div>
%         <div data-role="content" id="page1swipe" style="min-height:500px;">
%           <p>ここはページ1です。</p>
%           <p>>ボタンをクリックするとページ2に移動します。</p>
%         </div>
%       </div>
%
%       <!-- ページ2 -->
%       <div data-role="page" id="page2" data-theme="a">
%         <div data-role="header">
%           <!-- ナビゲーションバー(左右ページへのリンク) -->
%           <div data-role="navbar" class="navbar">
%             <ul>
%               <li>
%                 <a href="#page1" data-icon="arrow-l"></a>
%               </li>
%               <li>
%                 <a href="#page3" data-icon="arrow-r"></a>
%               </li>
%             </ul>
%           </div>
%         </div>
%         <div data-role="content" id="page2swipe" style="min-height:500px;">
%           <p>ここはページ2です。</p>
%           <p>>ボタンをクリックするとページ3に移動します。</p>
%           <p><ボタンをクリックするとページ1に移動します。</p>
%         </div>
%       </div>
%
%       <!-- ページ3 -->
%       <div data-role="page" id="page3" data-theme="a">
%         <div data-role="header">
%           <!-- ナビゲーションバー(左右ページへのリンク) -->
%           <div data-role="navbar" class="navbar">
%             <ul>
%               <li>
%                 <a href="#page2" data-icon="arrow-l"></a>
%               </li>
%             </ul>
%           </div>
%         </div>
%         <div data-role="content" id="page3swipe" style="min-height:500px;">
%           <p>ここはページ3です。</p>
%           <p><ボタンをクリックするとページ2に移動します。</p>
%         </div>
%       </div>
%       <!-- data-role="page" -->
%     </body>
%   </html>
% \end{lstlisting}
% \newpage

\end{document}
